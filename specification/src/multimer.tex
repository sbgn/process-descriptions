\subsection{Glyphs: \glyph{Multimers}}
\label{sec:multimer}

As its name implies, a multimer is an aggregation of multiple identical or pseudo-identical entities held together by non-covalent bonds (thus, they are distinguished from polymers by the fact that the later involve covalent bonds).
Here,  \emph{pseudo-identical} refers to the possibility that the entities differ chemically but retain some common global characteristic, such as a structure or function, and so can be considered identical within the context of the SBGN \PD.
An example of this is the homologous subunits in a hetero-oligomeric receptor.

SBGN \PD defines four different \glyph{multimer} glyphs: \glyph{simple chemical multimer}, \glyph{macromolecule multimer}, \glyph{nucleic acid feature multimer} and \glyph{complex multimer} as shown in \tab{multimer_containers}.

\begin{glyphDescription}

\glyphSboTerm
\begin{tabular}{l l}
    & SBO:0000286 ! multimer\\
\glyph{Simple chemical multimer} & SBO:0000421 ! multimer of simple chemicals\\
\glyph{Macromolecule multimer} & SBO:0000420 ! multimer of macromolecules \\
\glyph{Complex multimer} & SBO:0000418 ! multimer of complexes \\
\glyph{Nucleic acid feature multimer} & SBO:0000419 ! multimer of informational molecule segments \\
\end{tabular}


\glyphIncoming
Zero or more \glyph{production} arcs (\sect{production}).

\glyphOutgoing
Zero or more \glyph{consumption} arcs (\sect{consumption}), \glyph{modulation arcs} (\sect{modulations}), \glyph{logic arcs} (\sect{logicArc}), or \glyph{equivalence arcs} (\sect{equivalenceArc}).

\glyphContainer
Each type of \glyph{multimer} is represented by a different shape depending on the bio-molecular nature of its pseudo-identical subunits, as shown in \tab{multimer_containers}.
The shape of a \glyph{multimer} consists of two \glyph{subunits} or \glyph{EPNs} shapes shifted horizontally and vertically, and stacked on top of another.

\glyphLabel
A \glyph{multimer} is identified by a label that is a string of characters that may be distributed on several lines to improve readability.
The centre of the label must be placed on the centre of the shape.
The label may extend outside of the shape.
The label should refer to the pseudo-identical subunits, and not to the multimer itself.

\glyphAux A \glyph{multimer} may carry auxiliary units, depending on its type.

A \glyph{macromolecule}, \glyph{nucleic acid feature}, or \glyph{complex multimer} can carry one or more \glyph{state variables} that add information about its state (\sect{stateVariable}).
The state of such a \glyph{multimer} is defined as the list of all its \glyph{state variables}.

A \glyph{multimer} of any type can carry one or more \glyph{units of information} (\sect{unitInfo}).
These can characterize a domain, such as a binding site.
Particular \glyph{units of information} are available for describing the material type (\sect{material-types-cv}), conceptual type (\sect{conceptual-types-cv}) and the cardinality (\sect{cardinality-cv}) of such a \glyph{multimer}.

Note that a \glyph{state variable} or a \glyph{unit of information} carried by a \glyph{multimer} actually applies to each of the subunits individually.
If instead the \glyph{state variables} or the \glyph{units of information} are meant to apply to the whole multimeric assembly, a \glyph{macromolecule} (\sect{macromolecule}) or a \glyph{complex} (\sect{complex}) must be used instead of a \glyph{multimer}.
An assembly containing some \glyph{state variables} or \glyph{units of information} applicable to the subunits, and other \glyph{state variables} or \glyph{units of information} applicable to the assembly (for instance opening of a channel and phosphorylation of each of its subunits) must be represented by a \glyph{complex} (\sect{complex}).

Finally, a \glyph{simple chemical multimer} can also carry a \glyph{simple clone marker} (\sect{cloneMarker}), and a \glyph{macromolecule}, \glyph{nucleic acid feature} or \glyph{complex multimer} a \glyph{labelled clone marker} (\sect{cloneMarker}).

\end{glyphDescription}

\begin{table}[h]
\begin{tabu}{X[c,m]X[c,m]X[c,m]X[c,m]}
    \toprule
    \includegraphics[valign=m]{images/build/simple_chemical_multimer.pdf} & \includegraphics[valign=m]{images/build/macromolecule_multimer.pdf} & \includegraphics[valign=m]{images/build/genetic_multimer.pdf} & \includegraphics[valign=m]{images/build/complex_multimer.pdf}\\[0.5cm]
    \glyph{simple chemical multimer} & \glyph{macromolecule multimer} & \glyph{nucleic acid feature multimer} & \glyph{complex multimer}\\
	\bottomrule
\end{tabu}
\caption{The \PD glyphs for the different types of \glyph{multimers}.}
\label{tab:multimer_containers}
\end{table}
