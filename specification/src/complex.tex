\subsection{Glyph: \glyph{Complex}}\label{sec:complex}

A \glyph{complex} represents a pool of biochemical entities, each composed of other biochemical entities, whether macromolecules, simple chemicals, multimers, or other complexes. The resulting entity may have its own identity, properties and function in an SBGN map.
The \glyph{complex} can be described by the set of \glyph{subunits} (\sect{subunit}) it contains (see \fig{complexSubunits}). This description is entirely optional and is there to assist the user with a visual shorthand about the composition of the complex.

\begin{glyphDescription}

\glyphSboTerm
SBO:0000253 ! non-covalent complex

\glyphIncoming
Zero or more \glyph{production} arcs (\sect{production}).

\glyphOutgoing
Zero or more \glyph{consumption} arcs (\sect{consumption}), \glyph{modulation arcs} (\sect{modulations}), \glyph{logic arcs} (\sect{logicArc}), or \glyph{equivalence arcs} (\sect{equivalenceArc}).

\glyphContainer
A \glyph{complex} is represented by a rectangular shape with cut-corners (that is, an octagonal shape with sides of two different lengths).
If the \glyph{complex} is described by a set of \glyph{subunits}, then its shape must surround those of its \glyph{subunits}, and the size of the cut-corners must be adjusted so that there is no overlap between its shape and those of its \glyph{subunits}.
The shapes of the \glyph{subunits} must not overlap with each other or with the \glyph{clone marker} of the \glyph{complex} if it carries one.

\glyphLabel
A \glyph{complex} is identified by a label that is  a string of characters that may be distributed on several lines to improve readability.
In the case where the \glyph{complex} is not described by a set of \glyph{subunits}, the centre of the label must be placed on the centre of the \glyph{complex}'s shape.
In the case where the \glyph{complex} is described by a set of \glyph{subunits}, the label is optional and may be positioned to optimize the clarity and avoid overlapping, ideally between the bottom-most or the upper-most \glyph{subunit} and the border of the \glyph{complex}.

\glyphAux
A \glyph{complex} can carry one or more \glyph{state variables} that add information about its state (\sect{stateVariable}).

A \glyph{complex} can also carry one or more \glyph{units of information} (\sect{unitInfo}).
These can characterise a domain, such as a binding site.
Particular \glyph{units of information} are available for describing the material type (\sect{material-types-cv}) and the conceptual type (\sect{conceptual-types-cv}) of a \glyph{complex}.

A \glyph{complex} can also carry a \glyph{labelled clone marker} (see \sect{cloneMarker}).

Finally, as mentioned earlier, a \glyph{complex} can carry \glyph{subunits} in case it is described by a set of subunits.

\end{glyphDescription}

\begin{figure}[H]
  \centering
  \includegraphics{images/build/complex_combined.pdf}
  \caption{The \PD glyph for \glyph{complex}, shown plain and unadorned on the left, with an additional \glyph{state variable} and a \glyph{unit of information} in the middle, and with a \glyph{labelled clone marker} on the right.}
  \label{fig:complex}
\end{figure}
