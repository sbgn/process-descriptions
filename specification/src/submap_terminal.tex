\subsection{Glyph: \glyph{Submap terminal}}
\label{sec:submapTerminal}

A \glyph{submap teminal} is a decorator of the \glyph{submap} (\sect{submap}).
It is a named handle, or reference, to both an \glyph{EPN} (\sect{EPNs}) or \glyph{compartment} (\sect{compartment}) of the map, and a \glyph{tag} (\sect{tag}) of the map the \glyph{submap} glyph refers to.
Together with the \glyph{tag}, it allows linking glyphs of a map to their counterpart lying in a submap.

\begin{glyphDescription}

\glyphSboTerm Not applicable.

\glyphIncoming
One \glyph{equivalence arc} (\sect{equivalenceArc}).

\glyphOutgoing
None.

\glyphContainer A \glyph{submap terminal} is represented by a rectangular shape fused to an empty arrowhead, as shown in \fig{submapTerminal}.
The flat edge opposite to the arrowhead should be aligned to the edge of the \glyph{submap} glyph, and the incoming \glyph{equivalence arc} (\sect{equivalenceArc}) should be linked to its middle.

\glyphLabel A \glyph{submap terminal} is identified by a label that is  a string of characters that may be distributed on several lines to improve readability.
The centre of the label must be placed on the centre of the container.
The label may extend outside of the container.

\glyphAux
None.

\end{glyphDescription}

\begin{figure}[H]
  \centering
  \includegraphics{images/build/submap_terminal.pdf}
  \caption{The \PD glyph for \glyph{submap terminal}.}
  \label{fig:submapTerminal}
\end{figure}
