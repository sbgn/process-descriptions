\chapter{How to build a SBGN \PDm?}

It is important to realise that there is in general more than one way to represent a system in SBGN \PD. The choice of concepts and symbols often depend on the granularity of information available, and the message the authors of the map wish to convey to the readers of the map. 

% As an example, let's try to represent the phosphorylation of a multimeric protein.

%\begin{equation}
% 1 \times Tetramer_X + 2 \times ATP \longrightarrow 1 \times (PX)_2\_X + 2 \times ADP
%\end{equation}

% Example: Tetramer gets phosphorylated: 1 Tetramer (stoich 1), 2 ATP, 1 process, 1 enzyme, 1 complex with two subunit biphosphorylated, 

%\begin{equation}
% 1 \times Tetramer_X + 2 \times ATP \longrightarrow 1 \times (PX)_2\_X + 2 \times ADP
%\end{equation}
