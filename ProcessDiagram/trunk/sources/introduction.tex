% $HeadURL$

% =============================================================================
% introduction
% =============================================================================

\chapter{Introduction}

The goal of the \textbf{S}ystems \textbf{B}iology \textbf{G}raphical 
\textbf{N}otation (SBGN) is to standardize the graphical/visual 
representation of essential biochemical and cellular processes. SBGN 
defines comprehensive sets of symbols with precise semantics, together with 
detailed syntactic rules defining their use.  It also describes the manner 
xin which such graphical information should be interpreted. For a general 
description of SBGN, one can read:

\begin{quote}
 Nicolas Le Nov\`{e}re, Michael Hucka, Huaiyu Mi, Stuart Moodie, Falk 
Schreiber, Anatoly Sorokin, Emek Demir, Katja Wegner, Mirit I Aladjem, 
Sarala M Wimalaratne, Frank T Bergman, Ralph Gauges, Peter Ghazal, Hideya 
Kawaji, Lu Li, Yukiko Matsuoka, Alice Vill\'{e}ger, Sarah E Boyd, Laurence 
Calzone, Melanie Courtot, Ugur Dogrusoz, Tom C Freeman, Akira Funahashi, 
Samik Ghosh, Akiya Jouraku, Sohyoung Kim, Fedor Kolpakov, Augustin Luna, 
Sven Sahle, Esther Schmidt, Steven Watterson, Guanming Wu, Igor Goryanin, 
Douglas B Kell, Chris Sander, Herbert Sauro, Jacky L Snoep, Kurt Kohn  \& 
Hiroaki Kitano. The Systems Biology Graphical Notation. Nature 
Biotechnology 27, 735 - 741 (2009).  http://dx.doi.org/10.1038/nbt.1558
\end{quote}

This document defines the \emph{\PD{}} visual language of SBGN. \PDs are 
one of three views of a biological process offered by SBGN.  It is the 
product of many hours of discussion and development by many individuals and 
groups.

\section{SBGN levels and versions}
\label{sec:sbgn-levels}

It was clear at the outset of SBGN development that it would be impossible 
to design a perfect and complete notation right from the beginning.  Apart 
from the prescience this would require (which, sadly, none of the authors 
possess), it also would likely require a vast language that most newcomers 
would shun as being too complex.  Thus, the SBGN community followed an idea 
used in the development of other standards, i.e. stratify language 
development into levels.

A \emph{level} of one of the SBGN languages represents a set of features 
deemed to fit together cohesively, constituting a usable set of 
functionality that the user community agrees is sufficient for a reasonable 
set of tasks and goals.  Within \emph{levels}, \emph{versions} represent 
small evolution of a language, that may involve new glyphs, refined 
semantics, but no fundamental change of the way maps are to be generated 
and interpreted. Capabilities and features that cannot be agreed upon and 
are judged insufficiently critical to require inclusion in a given level, 
are postponed to a higher level or version.  In this way, the development 
of SBGN languages is envisioned to proceed in stages, with each higher 
levels adding richness compared to the levels below it.

\section{Developments, discussions, and notifications of updates}
\label{sec:discussions}

The SBGN website (\url{http://sbgn.org/}) is a portal for all things 
related to SBGN.  It provides a web forum interface to the SBGN discussion 
list (\mailto{sbgn-discuss@caltech.edu}) and information about how anyone 
may subscribe to it.  The easiest and best way to get involved in SBGN 
discussions is to join the mailing list and participate.

Face-to-face meetings of the SBGN community are announced on the website as 
well as the mailing list.  Although no set schedule currently exists for 
workshops and other meetings, we envision holding at least one public 
workshop per year.  As with other similar efforts, the workshops are likely 
to be held as satellite workshops of larger conferences, enabling attendees 
to use their international travel time and money more efficiently.

Notifications of updates to the SBGN specification are also broadcast on 
the mailing list and announced on the SBGN website.

\section{Note on typographical convention}
The concept represented by a glyph is written using a normal font, while a 
\glyph{glyph} means the SBGN visual representation of the concept. For 
instance ``a biological process is encoded by the SBGN PD \glyph{process}''.

% The following is for [X]Emacs users.  Please leave in place.
% Local Variables:
% TeX-master: "../sbgn_PD-level1"
% End:
