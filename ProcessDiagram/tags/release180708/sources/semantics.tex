\section{Semantic rules}

\subsection{Namespaces}

The notation has a concept of a namespace within which entities with the same 
identifying attributes are regarded as identical. The SBGN namespaces are shown 
in table \ref{tab:namespacedefs}.


\begin{center}
\tablecaption{Namespace scope definitions.}
\label{tab:namespacedefs}
\begin{small}
\tablefirsthead{\hline
  Namespace Scope & Entities affected & Notes\\\hline}
\tablehead{\hline
\multicolumn{3}{|l|}{\small\sl continued from previous page}\\
\hline\hline
  Namespace scope & Entities Effected & Notes\\\hline\hline}
\tabletail{\hline
\multicolumn{3}{|r|}{\small\sl continued on next page}\\
\hline}
\tablelasttail{\hline}
\begin{supertabular}{|l|p{5cm}|p{5cm}|}\hline
%
MapDiagram & CompartmentNode, SubMapDiagram, EquivalenceNode & \\\hline
%
CompartmentShape & BasicEntityNode & If no \glyph{compartment} is drawn then all BasicEntityNodes are assumed to belong to an invisible ``default'' compartment.\\\hline
EntityType & StateVariable, Annotation & \\\hline
ComplexType & EntityType & \\\hline
\end{supertabular}
\end{small}
\end{center}

\subsection{Cloning}

SBGN only allows identical nodes to duplicated on a map if they are
explicitly marked as such. This is using a \glyph{labeled clone marker} with all the
nodes that are identical. The details are shown in table \ref{tab:processduprules}.


\begin{center}
\tablecaption{Duplication rules.}
\label{tab:processduprules}
\begin{footnotesize}
\tablefirsthead{\hline
  Node & Can Duplicate? & Indication & Additional Rules\\\hline}
\tablehead{\hline
\multicolumn{4}{|l|}{\small\sl continued from previous page}\\
\hline\hline
  Node & Duplicate? & Indication & Additional Rules\\\hline\hline}
\tabletail{\hline
\multicolumn{4}{|r|}{\small\sl continued on next page}\\
\hline}
\tablelasttail{\hline}
\begin{supertabular}{|l|c|p{4cm}|p{3.5cm}|}\hline
%  Node & Duplicate? & How & Additional Rules\\\hline
CompartmentNode   & N & & \\\hline
SimpleChemicalNode & Y & \glyph{Simple clone marker} & \\\hline
UnspecifiedEntityNode & Y & \glyph{Simple clone marker} & \\\hline
SourceNode & N & & \\\hline
SinkNode & N & & \\\hline
Perturbation & Y & \glyph{Simple clone marker} & \\\hline
Observable & Y & \glyph{Simple clone marker} & \\\hline
MultimerChemicalEntity & Y & \glyph{Simple clone marker} & \\\hline
StatefulEntityPoolNode & Y & \glyph{Labeled clone marker} & \\\hline
MacromoleculeNode & Y & \glyph{Labeled clone marker} & \\\hline
MultimerMacromoleculeNode & Y & \glyph{Labeled clone marker} & \\\hline
Nucleic\-Acid\-FeatureNode & Y & \glyph{Labeled clone marker} & \\\hline
ComplexNode & Y & \glyph{Labeled clone marker} & \\\hline
TransitionNode & Y & None & Duplication is implied when all EPNs linked to the TransitionNode are marked as clones.\\\hline
OmittedProcessNode & Y & As Transition Node & \\\hline
UncertainProcessNode & Y & As Transition Node & \\\hline
AssociationNode & Y & As Transition Node & \\\hline
DissociationNode & Y & As Transition Node & \\\hline
LogicalOperatorNode & Y & None & \\\hline
ANDNode & Y & None & \\\hline
ORNode & Y & None & \\\hline
NOTNode & Y & None & \\\hline
\end{supertabular}
\end{footnotesize}
\end{center}

\subsection{State variables}

\glyph{States variables} are very simple. The variable can have a name. If the
name is set, it should be displayed. The variable can take any
value. The names used in the controlled vocabulary of
post-translational modification in section \ref{sec:CVs} are
reserved. Authors should avoid attaching any other meaning to them.

An entity pool node has a set of state variables, which is the union
of all the state variables used in the diagram. 
%(see diagram \ref{fig: sbgn}).  
All stateful EPNs must explicitly display the same set of
state variables and each state variable must be assigned a value or be
assigned the value ``Not Set''.

\subsection{Compartment spanning}

In all cases an EPN cannot \emph{belong} to more than one
compartment. However, an EPN can be \emph{drawn} over more than one
compartment. In such cases the decision on which is the owning
compartment is deferred to the drawing tool or the
author. ComplexNodes may contain EPNs which belong to different
compartments and in this way a complex can be used to describe an
entity that spans more than one compartment.

This restriction makes it impossible to represent in a semantically
correct way a macromolecule that spans more then one compartment ---
for example a receptor protein. It is clearly desirable to be able to
show a macromolecule in a manner that the biologist expects (i.e.\,
spanning from the outside through the membrane to the
inside). Therefore, the author is recommended to draw the
macromolecule across compartment boundaries, but the underlying SBGN
semantic model will assign it to only one. The assignment to a
compartment may be decided by the software drawing tool or the
author. Note that this has implications for auto-layout algorithms as
they will only be able to treat such entity nodes as contained within
a compartment and will have no way of knowing a macromolecule spans a
compartment.

The current solution is consistent with other Systems Biology
representations such as SBML and BioPAX. For more information about the
problems representing membrane spanning proteins and the rationale
behind the current solution see \sect{postponed}.

\subsection{Compartments}

The layout of compartments in an SBGN diagram does not imply anything
about the topology of compartments in the cell. Compartments should be
bounded and may overlap. However, adjacency and the nesting of
compartments does not imply that these compartments are next to each
other physically or hat one compartment contains the other.

\subsection{Modulation}

It is implied, but not defined explicitly that a process has a rate at
which it converts its input EPNs to its output EPNs. This concept is
important in understanding how SBGN describes process modulation.

\begin{enumerate}
\item A process with no modulations has an underlying ``basal rate''
  which describes the rate at which it converts inputs to outputs.
\item Modulation changes the basal rate in an unspecified fashion.
\item Stimulation is a modulation that's effect is to increase the basal rate.
\item Inhibition is a modulation that's effect is to decrease the basal rate.
\item The above types of modulation, when assigned to the same process are combined and have a multiplicative effect on the basal rate of the process.
\item Modulators that do not interact with each other in the above manner should be drawn as modulating different process nodes. Their effect is therefore additive.
\item At most one trigger can be assigned to a process. Two triggers
  would imply an implicit Boolean AND or OR operator. For clarity only
  one trigger can be assigned to a process and such combinations must be
  explicitly expressed as the Boolean operators.
\item At most one catalysis can be assigned to a process. A catalysis
  modulation implies that the exact biochemical mechanism underlying
  the process is known. In this context two catalysis reactions cannot
  be assigned to the same process as they are
  independent reactions. Other EPNs that modulate the catalysis can be
  assigned to the same process as modulators, stimulators, and
  inhibitors and will have a multiplicative modulation on the reaction
  rate defined by the catalysis.
\end{enumerate}

\subsection{Submaps}

Submaps are a visual device that allow a map to be split into several
views. They remain, however, part of the main map and share its
namespace. As a test of validity it should be possible to reintroduced
a submap into the main map by eliminating the SubMapNode and merging
the equivalent EntityPoolNodes in both maps.

\subsubsection{Rules for mapping to submaps}

An EntityPoolNode in the main map can be mapped to one in the submap
using a TagNode in the submap and SubMapTerminals (see \sect{submap}) in the main map. For a
mapping between map and submap to exists the following must be true:

\begin{enumerate}
\item The identifiers in the TagNode and SubMapTerminals must be identical.
\item The EntityPoolNodes must be identical.
\end{enumerate}

\subsubsection{Requirement to define a mapping}

If a map and submap both contain the same EntityPoolNode, then a
mapping between them must be defined as above.

\subsubsection{Cloning consistency}

If an EntityPoolNode is cloned in the main map then it must also be
marked as cloned in the submap even if there is only one copy of the
EntityPoolNode in the submap. The converse rule also applies where the
EntityPoolNode is cloned in the submap, but not the main map.
