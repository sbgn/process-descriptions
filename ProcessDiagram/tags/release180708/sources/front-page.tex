% $HeadURL$

\begin{titlepage}

\vspace*{0.75in}

\begin{center}

  \textbf{\sffamily\bfseries\huge
    Systems Biology Graphical Notation:\\[0.3em]
    Process Diagram Level 1}

\vspace*{0.5in}

\large
Draft of \sbgndate\\[0.25in]

\cornersize{0.3}\ovalbox{\begin{minipage}{4.9in}\color{DarkRed}
Disclaimer: This is a working draft of the SBGN Process Diagram
Level~1 specification.  It is not a normative document.
\end{minipage}}

\vspace{0.5in}

\textbf{\sffamily Editors}:\\[7pt]
\begin{tabular}{l>{\hspace*{15pt}}r}
Nicolas \lenov   & \emph{EMBL European Bioinformatics Institute, UK}\\
Stuart Moodie    & \emph{University of Edinburgh, UK}\\
Anatoly Sorokin  & \emph{University of Edinburgh, UK}\\
Michael Hucka	 & \emph{California Institute of Technology, USA}\\[7pt]
\end{tabular}

\vspace*{1em}
\textbf{\sffamily Principal Authors}:\\[7pt]
\begin{tabular}{l>{\hspace*{15pt}}r}
Nicolas \lenov   & \emph{EMBL European Bioinformatics Institute, UK}\\
Stuart Moodie    & \emph{University of Edinburgh, UK}\\
Anatoly Sorokin  & \emph{University of Edinburgh, UK}\\
Michael Hucka	 & \emph{California Institute of Technology, USA}\\
Falk Schreiber	 & \emph{IPK Gatersleben \& MLU Halle, Germany}\\
Emek Demir	 & \emph{MSKCC Computational Biology Center, USA}\\
Huaiyu Mi	 & \emph{SRI International, USA}\\
Yukiko Matsuoka	 & \emph{The Systems Biology Institute, Japan}\\
Katja Wegner	 & \emph{University of Hertfordshire, UK}\\
\multicolumn{2}{l}{\emph{and}}\\
Hiroaki Kitano	 & \emph{The Systems Biology Institute, Japan}\\
\end{tabular}

\vfill

\normalsize
\begin{minipage}{5in}
  \emph{To discuss any aspect of SBGN, please send your messages
    to the mailing list \mailto{sbgn-discuss@sbgn.org}.  To get
    subscribed to the mailing list or to contact us directly,
    please write to \mailto{sbgn-team@sbgn.org}.}
\end{minipage}

\vfill

\centerline{\includegraphics[width=1.25in]{\SBGNLogoFile}}

\end{center}

\end{titlepage}

% The title page is considered unnumbered and the next page after this
% starts with the page number 1 (actually, i), but the duplication of page
% number 1 confuses hyperref and leads to the following latex warning:
%
%   "pdfTeX warning (ext4): destination with the same identifier
%   (name{page.1}) has been already used, duplicate ignored"
%
% The following change makes the title page have page number 1 and the next
% page after that it becomes page ii.  This is unorthodox, but seems not
% completely unreasonable, and it avoids the confusing warning above.

\setcounter{page}{2}


% The following is for [X]Emacs users.  Please leave in place.
% Local Variables:
% TeX-master: "../sbgn_PD-level1"
% End:
