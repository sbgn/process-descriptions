% $HeadURL$

% ----------------------------------------------------------------
% Definitions that change with releases of the document.
% ----------------------------------------------------------------

\newcommand{\sbgndate}{31 August, 2009}

% ----------------------------------------------------------------
% Common commands when writing.
% ----------------------------------------------------------------

\newcommand{\SBGNPDLone}{SBGN \PD Level~1\xspace}

\newcommand{\PD}{Process Description\xspace}
\newcommand{\PDs}{Process Descriptions\xspace}
\newcommand{\PDm}{Process Description map\xspace}
\newcommand{\PDms}{Process Description maps\xspace}
\newcommand{\PDl}{Process Description language\xspace}

\newcommand{\ER}{Entity Relationship\xspace}
\newcommand{\ERs}{Entity Relationships\xspace}
\newcommand{\ERm}{Entity Relationship map\xspace}
\newcommand{\ERms}{Entity Relationship maps\xspace}
\newcommand{\ERl}{Entity Relationship language\xspace}

\newcommand{\AF}{Activity Flow\xspace}
\newcommand{\AFs}{Activity Flows\xspace}
\newcommand{\AFm}{Activity Flow map\xspace}
\newcommand{\AFms}{Activity Flow maps\xspace}
\newcommand{\AFl}{Activity Flow language\xspace}

% The name of a glyph is emphasized. Note that this is only the
% case when we talk about the glyph, not the biological concept
% represented by the glyph.

\newcommand{\glyph}[1]    {\emph{#1}} 
\newcommand{\vglyph}[1]   {\begin{sideways}\emph{#1}\end{sideways}}

\newcommand{\ERonly}      {\shabox{\textbf{ER only}}} 
\newcommand{\STonly}      {\shabox{\textbf{ST only}}} 
\newcommand{\AFonly}      {\shabox{\textbf{AF only}}} 
\newcommand{\NotER}       {\shabox{\textbf{Not ER. AF and ST only}}} 
\newcommand{\NotAF}       {\shabox{\textbf{Not AF. ST and ER only}}} 

\newcommand{\sbo}         {Systems Biology Ontology\xspace}
\newcommand{\sbourl}      {\url{http://www.ebi.ac.uk/sbo/}\xspace}
\newcommand{\sboid}[1]    {\texttt{#1}}

\newcommand{\lenov}       {Le~Nov\`{e}re\xspace}
\newcommand{\D}           {\displaystyle}
\newcommand{\tm}          {\textsuperscript{\tiny{\texttrademark}}}

\newcommand{\literalFont}[1]{\textup{\texttt{#1}}}
\newcommand{\figureFont}[1]{\textsf{\textbf{#1}}}

\newcommand{\link}[2]     {\literalFont{\href{#1}{#2}}}
\newcommand{\mailto}[1]   {\link{mailto:#1}{#1}}

\newcommand{\chap}[1]     {Chapter~\protect\ref{chp:#1}\xspace}
\newcommand{\sect}[1]     {Section~\protect\ref{sec:#1}\xspace}
\newcommand{\fig}[1]      {Figure~\protect\vref{fig:#1}\xspace}
\newcommand{\novreffig}[1]      {Figure~\protect\ref{fig:#1}\xspace}
\newcommand{\tab}[1]      {Table~\protect\vref{tab:#1}\xspace}
\newcommand{\eg}          {e.g.,\xspace}
\newcommand{\ie}          {i.e.,\xspace}

% \AddToShipoutPicture{\resizebox{0.9\pdfpagewidth}{0.9\pdfpageheight}% 
% {\rotatebox{60}{\color[gray]{0.8}\hspace*{5mm}\textsc{Working draft}}}}

% ----------------------------------------------------------------
% Commands for internal comments.
% ----------------------------------------------------------------

\newcommand\question[2]{{\color{red}[#1]}\marginpar{\color{red}\small\emph{#1: #2}}}
\newcommand\comment[2]{\footnote{{\color{blue}[#2] #1}}}

% Uncomment the definitions below to hide questions & comments.

% \renewcommand\question[2]{}
% \renewcommand\comment[2]{}


% ----------------------------------------------------------------
% SBGN logo.
% ----------------------------------------------------------------

\newcommand{\logofilebasename}{sbgn-logo-rgb}

\ifpdf
  % When creating PDFs, request the JPG format specifically,
  % because the resulting quality in the PDF final output is best
  % that way.
  \newcommand{\SBGNLogoFile}{\logofilebasename.jpg}
\else
  \newcommand{\SBGNLogoFile}{\logofilebasename.eps}
\fi

% Graphics path adjustments to get the logo.  The path setup is so
% that the \includegraphics can find the logo file no matter where
% the document is located (but obviously, it only works for
% certain path combinations -- it's a total hack).

\graphicspath{{.}{./images/}{../images/}}



% The following is for [X]Emacs users.  Please leave in place.
% Local Variables:
% TeX-master: "../sbgn_PD-level1"
% End:
